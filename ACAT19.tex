\documentclass[a4paper]{jpconf}


\address{$ˆ1$ Mathematics Faculty, Open University,
Milton Keynes MK7 ̃6AA, UK}
\address{$ˆ2$ Department of Mathematics,
Imperial College, Prince Consort Road, London SW7 ̃2BZ, UK}
\address{$ˆ3$ Department of Computer Science,
University College London, Gower Street, London WC1E ̃6BT, UK}
\ead{williams@ucl.ac.uk}
\begin{abstract}
The abstract appears here.
\end{abstract}



\title{Making RooFit Ready for Run3}
\author{S Hageb\"ock$^1$ and L Moneta$^2$}
\address{$^1$ TODO CERN}
\ead{stephan.hageboeck@cern.ch}

\begin{abstract}
RooFit and RooStats, the toolkits for statistical modelling in ROOT, are used in most searches and measurements at the Large Hadron Collider. The data to be collected in Run 3 will enable measurements with higher precision and models with larger complexity, but also require faster data processing.

In this talk, first results on vectorising and multi-threading likelihood fits in RooFit will be presented. These improvements will enable the LHC experiments to process larger datasets without having to compromise with respect to model complexity.

\end{abstract}

\section[Test1]{Test1}
Test text.


\section*{References}
TODO
